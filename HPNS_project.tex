\documentclass[12pt]{article}

\usepackage[margin=1in]{geometry} 
\usepackage{amsmath,amsthm,amssymb,amsfonts, mathrsfs, bbm, mathtools}
\usepackage{enumitem, graphicx}
 
\newcommand{\N}{\mathbb{N}}
\newcommand{\Z}{\mathbb{Z}}
\newcommand{\R}{\mathbb{R}}
 
\newenvironment{excercise}[2][Excercise]{\begin{trivlist}
\item[\hskip \labelsep {\bfseries #1}\hskip \labelsep {\bfseries #2.}]}{\end{trivlist}}

%If you want to title your bold things something different just make another thing exactly like this but replace "problem" with the name of the thing you want, like theorem or lemma or whatever
 
\begin{document}
 
%\renewcommand{\qedsymbol}{\filledbox}
%Good resources for looking up how to do stuff:
%Binary operators: http://www.access2science.com/latex/Binary.html
%General help: http://en.wikibooks.org/wiki/LaTeX/Mathematics
%Or just google stuff
 
\title{On the random behavior of asynchronous computing}
\author{Huei-Syuan Yan, Cherng-En Lee, Chi-Hao Wu}
\maketitle

\section{Introduction}
   In this project, we investigate the random behavior of asynchronuous computing. Given a system $Ax = b$, an iterative method solve it by letting
   \[
      x^{(k+1)} = T x^{(k)} + c \text{,}
   \]
   where $T$ is an iteration matrix, and $c$ is a constant vector. \\[5pt]
   According to \cite{C-M}, asynchronous method is guaranteed to converge if and only if the spectrum, $\sigma(|T|)$, is strictly less than one, which is different from the criterion of synchronous computing scheme, $\sigma(T) < 1$. However, it does not imply that an asynchronous method will always diverge when $\sigma(|T|) \geq 1$. \\[5pt]
   An interesting question then arises naturally. How likely is an asychronous scheme going to converge when its iteration matrix does not satisfy the crierion in \cite{C-M}?

\section{Main Result}
   Consider $A_{\alpha} = \alpha^{-1} D + R$. The iteration matrix for solving $A_{\alpha} x = b$ with Jacobi's method is
   \[
      T_{\alpha} = - (\alpha^{-1}D)^{-1}R = -\alpha D^{-1}R \quad\text{where $\alpha > 0$.}
   \]
   We can see that the parameter $\alpha$ controlls the spectrums of $|T_{\alpha}|$; the spectrums increase as $\alpha$ increases. In addition, $\alpha$ also affects the diagonal of $A_{\alpha}$; the magnitude of iagonals decrease as $\alpha$ increases. \\[5pt]
   \begin{enumerate}[label=(\roman*)]
   \item
   \[
      A_{\alpha} = 
      \begin{bmatrix}
         7/\alpha &  1 &  2 &  3 & -1\\
         3 & 13/\alpha & -2 & -1 & -7\\
         2 &  1 &  5/\alpha & -2 &  0\\
         2 & -1 &  5 &  11/\alpha & -3\\
        -2 & -3 & -1 &  1 &  7/\alpha
      \end{bmatrix}
   \]
   \item
   \[
      A_{\alpha} = 
      \begin{bmatrix}
         3/\alpha &  -1 & -1 &  1 & 0\\
         2 & 17/\alpha & -5 & 6 & -4\\
         1 &  2 &  5/\alpha & -1 & -1\\
         -3 & -5 & -7 &  19/\alpha & -4\\
        -3 & 1 & -2 &  1 &  7/\alpha
      \end{bmatrix}
   \]
   \item
   \[
      A_{\alpha} = 
      \begin{bmatrix}
         23/\alpha &  2 & -11 & -9 & -1\\
         9 & 17/\alpha & 1 & -1 & 6\\
         12 &  -2 &  29/\alpha & 3 & -12\\
         -29 & 1 & 0 &  31/\alpha & 1\\
        -1 & 1 & 1 &  34 &  37/\alpha
      \end{bmatrix}
   \]
   \end{enumerate}
\section{Conclusion}  
   According these results, a less diagonally dominated $A_{\alpha}$ should perform relatively poor using asynchronous scheme; this property is very similar to the Gauss-Seidel's method. In fact, we can consider Gauss-Seidel's method as an asynchronous computing process with a very \emph{regular} pattern.

\begin{thebibliography}{2}
\bibitem{C-M}
D. Chazan, W. Miranker. 
\textit{Chaotic relaxation.} 
Linear Algebra and its Applications, Vol 2, Issue 2, April 1969, Pages 199-222.
\bibitem{B}
G.M. Baudet. 
\textit{Asynchronous Iterative Methods for Multiprocessors} 
Journal of the ACM, Vol 25, Issue 2, April 1978, Pages 226-244.
\end{thebibliography}

\end{document}
